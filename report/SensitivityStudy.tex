\documentclass{article}

\usepackage{graphicx} % Required for the inclusion of images
\usepackage{natbib} % Required to change bibliography style to APA
\usepackage{amsmath} % Math package
%\setlength\parindent{0pt} Can remove paragraph indentation

\renewcommand{\labelenumi}{\alph{enumi}.} % Make numbering in the enumerate environment by letter rather than number (e.g. section 6)

%\usepackage{times} % Uncomment to use the Times New Roman font

%------------------------------------------------------------------------
%	Begin Document
%------------------------------------------------------------------------

\title{WATCHMAN Sensitivity to Reactor Power Statuses at the Boulby and
Fairport Sites.}
    
\author{Teal Pershing}

\date{\today}

\begin{document}

\maketitle % Inserts the title, author, and date

% If you wish to include an abstract, uncomment the lines below
% \begin{abstract}
This report provides details on a study to determine the expected WATCHMAN
sensitivity to reactor antineutrinos when reactors are on or off.  The study
was performed for the Proposed Boulby and Hartlepool sites. It is assumed that
the exact reactor outage schedules were known.  For the Boubly site, confirmation
of the two neighboring reactors turning on and off by measuring the IBD event 
rate is expected in
425 days of running (68.3\% CL).  For the Fairport site, confirmation of the one
neighboring reactor turning on and off is expected within 74 days of running (68.3\% CL).
% \end{abstract}

%----------------------------------------------------------------------------------------
%	SECTION 1
%----------------------------------------------------------------------------------------

\section{Introduction}

Principle of operation for water-gadolinium doped detectors is briefly discussed.
We then provide an overview of the WATer CHerenkov Monitor of ANtineutrinos 
(WATCHMAN) detector.  A brief summary of the difference in each proposed detector
site is presented.

\subsection{Reactor Antineutrinos}

Need to wrap antineutrinos into how we want to search for smaller reactors.  
Measuring the antineutrino flux from a reactor as a means of discovering rogue
reactors or monitoring a reactor's power cycles is powerful for two reasons:

\item{Shielding cannot stop antineutrinos and prevent observation}
\item{Antineutrinos interact weakly, and as such travel far.  This provides
    opportunities for reactor monitoring/discovery at a distance.}

Reactor antineutrinos are antineutrinos emitted both during the fission process
and in the beta decays of isotopes left over following a fission. 
Note that the reactor type (i.e. the reactor fuel composition used by the
reactor type) and operating thermal power are the key factors that vary
the antineutrino flux.

%The antineutrino flux is determined almost completely by the fissions of 
%$^{235}U$, $^{238}U$, $^{239}Pu$, and $^{241}Pu$.

\subsection{Detector Overview}

The WATCHMAN


%----------------------------------------------------------------------------------------
%	SECTION 2
%----------------------------------------------------------------------------------------

\section{Experimental Data}

\begin{tabular}{ll}
EHHEHEHE
\end{tabular}

%----------------------------------------------------------------------------------------
%	SECTION 3
%----------------------------------------------------------------------------------------

\section{Sample Calculation}


%----------------------------------------------------------------------------------------
%	SECTION 4
%----------------------------------------------------------------------------------------

\section{Results and Conclusions}

The atomic weight of magnesium is concluded to be , as determined by the stoichiometry of its chemical combination with oxygen. This result is in agreement with the accepted value.

\begin{figure}[h]
\begin{center}
\includegraphics[width=0.65\textwidth]{placeholder} % Include the image placeholder.png
\caption{Figure caption.}
\end{center}
\end{figure}

%----------------------------------------------------------------------------------------
%	SECTION 5
%----------------------------------------------------------------------------------------

\section{Discussion of Experimental Uncertainty}

The accepted value (periodic table) is  \cite{Smith:2012qr}. The percentage discrepancy between the accepted value and the result obtained here is 1.3\%. Because only a single measurement was made, it is not possible to calculate an estimated standard deviation.

The most obvious source of experimental uncertainty is the limited precision of the balance. Other potential sources of experimental uncertainty are: the reaction might not be complete; if not enough time was allowed for total oxidation, less than complete oxidation of the magnesium might have, in part, reacted with nitrogen in the air (incorrect reaction); the magnesium oxide might have absorbed water from the air, and thus weigh ``too much." Because the result obtained is close to the accepted value it is possible that some of these experimental uncertainties have fortuitously cancelled one another.

%----------------------------------------------------------------------------------------
%	SECTION 6
%----------------------------------------------------------------------------------------

\section{Answers to Definitions}

\begin{enumerate}
\begin{item}
The \emph{atomic weight of an element} is the relative weight of one of its atoms compared to C-12 with a weight of 12.0000000$\ldots$, hydrogen with a weight of 1.008, to oxygen with a weight of 16.00. Atomic weight is also the average weight of all the atoms of that element as they occur in nature.
\end{item}
\begin{item}
The \emph{units of atomic weight} are two-fold, with an identical numerical value. They are g/mole of atoms (or just g/mol) or amu/atom.
\end{item}
\begin{item}
\emph{Percentage discrepancy} between an accepted (literature) value and an experimental value is
\begin{equation*}
\frac{\mathrm{experimental\;result} - \mathrm{accepted\;result}}{\mathrm{accepted\;result}}
\end{equation*}
\end{item}
\end{enumerate}

%----------------------------------------------------------------------------------------
%	BIBLIOGRAPHY
%----------------------------------------------------------------------------------------

\bibliographystyle{apalike}

\bibliography{sample}

%----------------------------------------------------------------------------------------


\end{document}
