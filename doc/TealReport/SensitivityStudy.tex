\documentclass{article}

\usepackage{graphicx} % Required for the inclusion of images
\usepackage{natbib} % Required to change bibliography style to APA
\usepackage{amsmath} % Math package
%\setlength\parindent{0pt} Can remove paragraph indentation

\renewcommand{\labelenumi}{\alph{enumi}.} % Make numbering in the enumerate environment by letter rather than number (e.g. section 6)

%\usepackage{times} % Uncomment to use the Times New Roman font

%------------------------------------------------------------------------
%	Begin Document
%------------------------------------------------------------------------

\title{WATCHMAN Sensitivity to Reactor Power Statuses at the Boulby and
Fairport Sites.}
    
\author{Teal Pershing}

\date{\today}

\begin{document}

\maketitle % Inserts the title, author, and date

% If you wish to include an abstract, uncomment the lines below
% \begin{abstract}
This report provides details on a study to determine the expected WATCHMAN
sensitivity to reactor antineutrinos when reactors are on or off.  The study
was performed for the Proposed Boulby and Hartlepool sites. It is assumed that
the exact reactor outage schedules were known.  For the Boubly site, confirmation
of the two neighboring reactors turning on and off by measuring the IBD event 
rate is expected in
less than 600 days of running (95\% CL).  For the Fairport site, confirmation of the one
neighboring reactor turning on and off is expected within 4 days of the first
change of power state (95\% CL).
% \end{abstract}

%----------------------------------------------------------------------------------------
%	SECTION 1
%----------------------------------------------------------------------------------------

\section{Introduction}

Principle of operation for water-gadolinium doped detectors is briefly discussed.
We then provide an overview of the WATer CHerenkov Monitor of ANtineutrinos 
(WATCHMAN) detector.  A brief summary of the difference in each proposed detector
site is presented.

\subsection{Reactor antineutrinos}

Antineutrinos are constantly emitted from operating nuclear reactor complexes.
As nuclear reactors convert the energy release from nuclear fissions 
to electrical power, antineutrinos are emitted in the beta decays of 
isotopes left over following a fission. For each fission, an average of six
antineutrinos will be released in the decay of daugher isotopes.  The thermal
power output of a reactor is directly proportional to the total antineutrino
flux emitted from a reactor.  A $3GW_{t}$ reactor will typically emit on the
order of $10^{20}$ antineutrinos per second.

Note that the reactor type (as well as the reactor fuel composition used by the
reactor type) and operating thermal power are the key factors that vary
the antineutrino flux. Commonly deployed nuclear reactor models in the modern 
day include pressurized water reactors (PWR), boiling water reactors (BWR),
CANDU reactors, and advanced gas-cooled reactors (AGR).  The water reactors
employ light water as their moderating medium.  Different reactors have
different fueling cycles and maintenance requirements.  For all of these
nuclear reactors, the antineutrino flux is determined almost completely by 
the isotopic abundance (and fissioning) of $^{235}U$, $^{238}U$, 
$^{239}Pu$, and $^{241}Pu$.


Measuring the antineutrino flux from a reactor as a means of discovering rogue
reactors or monitoring a reactor's power cycles is powerful for two reasons:
\begin{itemize}
\item{Shielding cannot stop antineutrinos and prevent observation}
\item{Antineutrinos interact weakly, and as such travel far.  This provides
    opportunities for reactor monitoring/discovery at a distance.}
\end{itemize}

\subsection{The inverse beta decay}
Antineutrinos emitted from reactors are primarily detected using a delayed
coincidence signal produced by the inverse beta decay (IBD) interaction.  
The IBD interaction proceeds as follows:

$\nu_{e} \, + \, p \, \rightarrow \, e^{+} \, + \, n$

In a water Cherenkov detector, the antineutrino will generally interact with
a hydrogen proton in water.  Prompt light is visible in the detector from the
positron's Cherenkov light and eventual annihilation with a positron.  After
some time (~typically tens/hundreds of microseconds), the neutron produced
will thermalize and be captured, producing the delayed light signal.

%FIXME: ADD THAT NICE IBD FIGURE I USED IN THE POSTER?

In pure water, the free neutron is typically captured by another proton.  This
capture forms an excited deuteron, which will de-excite and emit a 2.2 MeV
gamma.  The gamma will compton scatter off of electrons in the water; these
electrons can emit Cherenkov light visible in the detector.  Unfortunately,
a combination of this energy lying within many natural background signals 
(naturally occuring radioactivity present in the water, PMTs, etc.) and the
large energy resolution (~hundreds of keV) reduces the detection efficiency of
this delayed signal considerably.

The WATCHMAN detector will use gadolinium-doped water to increase the delayed
signal detection efficiency.  Gadolinium boosts the coincidence trigger
efficiency in two ways:
\begin{enumerate}
    \item Gadolinium has a large thermal neutron cross-section. This results
        in a faster neutron capture time (\~20-30 microseconds) than pure water.
    \item After thermal capture, gadolinium (specifically $^{155}Gd$ and 
        $^{157}Gd$ de-excites and emits 3-4 gammas for a total of 7-8 MeV
        energy release.
\end{enumerate}

A Gadolinium source was developed by the SuperKamiokande collaboration and
deployed in SuperK to determine the energy profile of gadolinium de-excitation
in ultrapure water (SOURCE THIS).  The average energy detected was 4.3 MeV,
a considerable energy increase (see figure AAH).
%FIXME: ADD THE SUPERK FIGURE ABOVE AND SOURCE THIS MAN

\subsection{Detector overview}

The WATCHMAN demonstration is a proposed Gadolinium-doped water Cherenkov
detector.  ~0.1\% Gadolinium would be loaded into ultrapure water, with a 
final detection volume mass of 1 kiloton.  WATCHMAN would utilize ~3000
high quantum efficiency,low radioactivity photomultiplier tubes (PMTs) to
detect the IBD events occuring in the detection volume.

\subsubsection{Detector location}
The WATCHMAN deployment location has been narrowed down to two locations.
The first location would be the IMB mine in Painesville, Ohio.  The IMB
mine is 13 kilometers from the Perry Reactor Facility.  The Perry Reactor
Facility has one core with a operating thermal power of 3875 MW. The reactor
is a pressurized water reactor. 
The second location would be the Boulby underground science laboratory in
Boulby, England.  The boulby site is 25 kilometers from the Hartlepool
reactor facility.  The Hartlepool reactor facility has two reactor cores
that each operate at 1500 MW.  The reactors are advanced gas-cooled reactors.
%FIXME: INCLUDE FIGURE FROM TALKS WITH OVERVIEWS OF EACH SITE AND WATCHMAN?


\section{Sensitivity study overview}

With two sites in consideration for deployment, an estimation of the 
detector's success in reactor monitoring was needed for each site.
As a first figure of merit, the following scenario was considered:

\vspace{1cm}
\textit{Assume that we know the exact outage schedule of both cores at the 
    Heartlepool reactor near Boulby, or Perry’s one reactor core near
    Fairport.  For each site, how many days would WATCHMAN need to 
    operate to see a clear 3σ difference between the “core(s) on” data
    and “one core off” data?}
\vspace{1cm}

This study was meant as a first step in comparing the detection capability
at each site.  Further important studies would include evaluating
WATCHMAN's sensitivity to reactor on and off states with no prior
schedule knowledge.

The study was completed in two main stages:
\begin{enumerate}
    \item For each site, use RAT-PAC to simulate 
        the average IBD candidates/day expected in WATCHMAN
        due to signal sources (the neighboring reactors) and background
        sources (other reactors, accidentals, fast neutrons, and
        radionuclides).
    \item Using the averages output from RAT-PAC, statistically generate
        many experiments using the expected event rates at each site.  
        For each experiment, determine the day at which a 3-sigma difference
        is seen between the event rate in the "cores on" days data set and
        the "one core off" days data set.
\end{enumerate}

\subsection{RAT-PAC generated event rates}
This document focuses on the software used to statistically generate the
WATCHMAN experiments.  For more information on RAT-PAC and how the 
average IBD event rates are simulated, see here.
%FIXME: ASK MARC BERGEVIN FOR A SOURCE TO POINT TO.

The average IBD rates output from RAT-PAC for each deployment option
(and used for the results in the statistically generated experiments)
can be found below.  Note that the photocoverage is chosen as 20\% for the
IMB mine deployment and 25\% for the Boulby mine deployment.  These are
the proposed photocoverages for each site as of July 2017.
%FIXME: ADD THE JSON FILES THAT HAVE THE EVENT RATES FOR EACH PC.

\subsection{Statistical generation of WATCHMAN experiments}

A program was written in python to utilize the RAT-PAC event rates to produce
statistically generate experiments. The program can be found at the following
github repository:

\vspace{1cm}
https://github.com/pershint/BoulbySignalAnalysis.git
\vspace{1cm}

See the README included for tips on adjusting the experimental parameters 
(experimental photocoverage, reactor schedule, addition of maintenance outages,
etc.).

When a single statistical experiment is generated, the following steps are
executed in the code:

\begin{itemize}
    \item For the selected photocoverage and experiment location, the average
        IBD rates for each signal & background source are acquired from the
        local database (found in the DB directory).
    \item For a single day in the experiment, a # IBD events due to each signal 
        and background source is determined by randomly sampling from a Poisson
        distribution with the average acquired from the database. The
        neighboring reactors are assumed to be on for every day.
    \item The above step is repeated for every day in the experiment.
    \item The reactor outage schedule selected at script start is then used to
        set the neighboring reactor IBD events to 0 on days where the reactor
        is turned off.
\end{itemize}

Graphing tools to see the output from a generated experiment can be found in
the "graph" directory of the main repository.

\subsection{Extraction of WATCHMAN sensitivity from generated experiments}

The expected WATCHMAN sensitivity to a reactor's on\/off schedule is 
determined by generating many experiments as described above.  For each
individual generated experiment, the day at which a clear three-sigma difference
between the IBD rate in the data set where all neighboring cores are on and
the IBD rate in the data where at least one neighboring core is off is found.
Then, the distribution of these determination
days is plotted and used to provide confidence limits regarding how quickly
WATCHMAN can confirm a reactor's operating schedule.

\subsubsection{Finding the day of schedule confirmation
(i.e. determination day)}

For a generated experiment, the data are first collected into two subsets. One
set contains the IBD candidates that occured on days where all neighboring
reactors were on.  The second set contains the #IBD events that occured
on days where at least one neighboring reactor is off.

Consider the data set where all neighboring reactors are on.  For a given day
$D$ in the experiment, the average IBD rate observed in this data set on
day $D$ will be:

$$\Gamma_{on} = \frac{N_{on}}{t_{on}}$$

Where $N_{on}$ is the total number of IBD candidates in the "on" data set, and
$t_{on}$ is the number of days all neighboring reactors are on up to day $D$.
The uncertainty on this average (assuming no error in the number of days) will
then be:

$$\sigma_{\Gamma_{on}} = \frac{\sqrt{N_{on}}}{t_{on}} = \sqrt{\frac{
    \Gamma_{on}}{t_{on}} $$

The same average and uncertainty can be calculated for the data set where at
least one neighboring reactor is off.

To find the schedule confirmation day, the difference between the "on" and "off"
data set described above is calculated for each day.  The total error on this
difference is given by:

$$\sigma_{tot} = \sqrt{(\sigma_{\Gamma_{on}})^{2} + (\sigma_{\Gamma_{on}})^{2}}$$

The schedule confirmation day is the first day at which the following relation
is true, and remains true for 14 days:

$$\Gamma_{on} - \Gamma_{off} > 3 \sigma_{tot}$$

An example of a statistically generated experiment showing $\Gamma_{on}$, 
$\Gamma_{off}$, and the difference of the two can be found below.

%FIXME: ADD FIGURES WITH AN EXAMPLE OF FINDING THE DETERMINATION DAY.

\subsubsection{Confidence limits using generated determination days}

Many statistically generated experiments are used to determine the
confidence limit for when WATCHMAN would confirm a clear separation in
 the "reactors on" and "at least one reactor off" data.  The determination
 days are placed into a histogram giving the determination day distribution
 as a function of day in the experiment.  Using the cumulative sum of the
 distribution, the 68.3\%, 95\%, and 99.7\% CLs can be determined. Examples
 of the determination day distribution and associated CL graph can be
 found below.

%FIXME: ADD IN THE CL AND PURPLE HISTOGRAM EXAMPLES AS FIGURES

 One must be careful to generate enough experiments such that the tail of
 the determination day distribution is populated with a decent number of
 events.  A lack of determination days in the tail may result in a large
 error on the CLs drawn (especially in the 99.7\% CL value).

\section{Sensitivity study results}

The expected sensitivity for the WATCHMAN detector at the Boulby and Fairport
deployment options are shown below.  The photocoverage for the Boulby and
Fairport configuration are 25\% and 20\%, respectively.

\subsection{Fairport sensitivity}
%FIXME: Need the plots that show the determination day starting right when
%WATCHMAN turns off, and when it turns off 40 days in
Discuss how the determination happens within several days of the detector's
state changing.  Again, this result only confirms when we would see a clear
difference in the "on" and "one reactor off" data set assuming the complex
follows the given schedule.  The speed at which confirmation is accomplished,
however, is a strong indicator that any single power state change lasting
even a week long will likely be visible.  A follow-up study focusing on the
sensitivity to single power-state changes would be necessary to confirm this.

\


%----------------------------------------------------------------------------------------
%	SECTION 2
%----------------------------------------------------------------------------------------

\section{Experimental Data}

\begin{tabular}{ll}
EHHEHEHE
\end{tabular}

%----------------------------------------------------------------------------------------
%	SECTION 3
%----------------------------------------------------------------------------------------

\section{Sample Calculation}


%----------------------------------------------------------------------------------------
%	SECTION 4
%----------------------------------------------------------------------------------------

\section{Results and Conclusions}

The atomic weight of magnesium is concluded to be , as determined by the stoichiometry of its chemical combination with oxygen. This result is in agreement with the accepted value.

\begin{figure}[h]
\begin{center}
\includegraphics[width=0.65\textwidth]{placeholder} % Include the image placeholder.png
\caption{Figure caption.}
\end{center}
\end{figure}

%----------------------------------------------------------------------------------------
%	SECTION 5
%----------------------------------------------------------------------------------------

\section{Discussion of Experimental Uncertainty}

The accepted value (periodic table) is  \cite{Smith:2012qr}. The percentage discrepancy between the accepted value and the result obtained here is 1.3\%. Because only a single measurement was made, it is not possible to calculate an estimated standard deviation.

The most obvious source of experimental uncertainty is the limited precision of the balance. Other potential sources of experimental uncertainty are: the reaction might not be complete; if not enough time was allowed for total oxidation, less than complete oxidation of the magnesium might have, in part, reacted with nitrogen in the air (incorrect reaction); the magnesium oxide might have absorbed water from the air, and thus weigh ``too much." Because the result obtained is close to the accepted value it is possible that some of these experimental uncertainties have fortuitously cancelled one another.

%----------------------------------------------------------------------------------------
%	SECTION 6
%----------------------------------------------------------------------------------------

\section{Answers to Definitions}

\begin{enumerate}
\begin{item}
The \emph{atomic weight of an element} is the relative weight of one of its atoms compared to C-12 with a weight of 12.0000000$\ldots$, hydrogen with a weight of 1.008, to oxygen with a weight of 16.00. Atomic weight is also the average weight of all the atoms of that element as they occur in nature.
\end{item}
\begin{item}
The \emph{units of atomic weight} are two-fold, with an identical numerical value. They are g/mole of atoms (or just g/mol) or amu/atom.
\end{item}
\begin{item}
\emph{Percentage discrepancy} between an accepted (literature) value and an experimental value is
\begin{equation*}
\frac{\mathrm{experimental\;result} - \mathrm{accepted\;result}}{\mathrm{accepted\;result}}
\end{equation*}
\end{item}
\end{enumerate}

%----------------------------------------------------------------------------------------
%	BIBLIOGRAPHY
%----------------------------------------------------------------------------------------

\bibliographystyle{apalike}

\bibliography{sample}

%----------------------------------------------------------------------------------------


\end{document}
